\documentclass[titlepage, fleqn, a4paper, 12pt, twoside]{article}
\usepackage{geometry}
\usepackage{exsheets} %question and solution environments
\usepackage{amsmath, amssymb, amsthm} %standard AMS packages
\usepackage[utf8]{inputenc}
\usepackage{esint} %integral signs
\usepackage{marginnote} %marginnotes
\usepackage{gensymb} %miscellaneous symbols
\usepackage{commath} %differential symbols
\usepackage{xcolor} %colours
\usepackage{cancel} %cancelling terms
\usepackage[free-standing-units,space-before-unit]{siunitx} %formatting units
	\sisetup
	{
		per-mode=fraction,
		fraction-function=\frac
	}
\usepackage{tikz, pgfplots} %diagrams
	\usetikzlibrary{calc, hobby, patterns, intersections, angles, quotes, spy}
	\usetikzlibrary{circuits.logic.US,circuits.logic.IEC}
\usepackage{graphicx} %inserting graphics
\usepackage{imakeidx}
\makeindex
\usepackage{hyperref} %hyperlinks
\usepackage{datetime} %date and time
\usepackage{enumerate, enumitem} %numbered lists
\usepackage{float} %inserting floats
\usepackage[american voltages]{circuitikz} %circuit diagrams
\usepackage{setspace} %double spacing
\usepackage{microtype} %micro-typography
\usepackage{listings} %formatting code
	\lstset{language=Matlab}
	\lstdefinestyle{standardMatlab}
	{
		belowcaptionskip=1\baselineskip,
		breaklines=true,
		frame=L,
		xleftmargin=\parindent,
		language=C,
		showstringspaces=false,
		basicstyle=\footnotesize\ttfamily,
		keywordstyle=\bfseries\color{green!40!black},
		commentstyle=\itshape\color{purple!40!black},
		identifierstyle=\color{blue},
		stringstyle=\color{orange},
	}
\usepackage{booktabs}
\usepackage{multirow}
\usepackage{todonotes}
\usepackage[noabbrev,capitalize]{cleveref}
\usepackage[section]{placeins}
\usepackage[style=numeric, backend=biber]{biblatex}
\usepackage{adjustbox}
\usepackage{algpseudocode} %algorithms
\usepackage{algorithm} %algorithms

% \bibliography{<mybibfile>}% ONLY selects .bib file; syntax for version <= 1.1b
\addbibresource{bibliography.bib}% Syntax for version >= 1.2

\newcommand\numberthis{\addtocounter{equation}{1}\tag{\theequation}} %adds numbers to specific equations in non-numbered list of equations

\DeclareMathAlphabet{\mathcal}{OT1}{pzc}{m}{it}

\theoremstyle{definition}
\newtheorem{example}{Example}
\newtheorem{definition}{Definition}

\theoremstyle{theorem}
\newtheorem{theorem}{Theorem}
\newtheorem{law}{Law}

\tikzset{
block/.style = {draw, rectangle, minimum height=3em, minimum width=3em},
tmp/.style  = {coordinate},
sum/.style = {draw, circle, node distance=1cm},
input/.style = {coordinate},
output/.style = {coordinate},
pinstyle/.style = {pin edge={to-,thin,black}}
}

\makeatletter
\@addtoreset{section}{part} %resets section numbers in new part
\makeatother

\newcommand\blfootnote[1]{%
	\begingroup
	\renewcommand\thefootnote{}\footnote{#1}%
	\addtocounter{footnote}{-1}%
	\endgroup
}

\renewcommand{\marginfont}{\scriptsize \color{blue}}

\renewcommand{\tilde}{\widetilde}

\def\doubleunderline#1{\underline{\underline{#1}}}

\SetupExSheets{solution/print = true} %prints all solutions by default

\DeclareMathOperator{\FT}{\mathcal{F}}
\DeclareMathOperator{\IFT}{\mathcal{F}^{-1}}
\DeclareMathOperator{\DFT}{\mathcal{DFT}}

\DeclareMathOperator{\Q}{\mathcal{Q}}

\DeclareMathOperator{\rect}{\mathrm{rect}}
\DeclareMathOperator{\sinc}{\mathrm{sinc}}

\DeclareMathOperator{\boxfunc}{\mathrm{box}}

\DeclareMathOperator{\vspan}{\mathrm{span}}

\DeclareMathOperator{\argmin}{\mathrm{argmin}}

\DeclareMathOperator{\Db}{\mathrm{Db}}

\DeclareMathOperator{\normal}{\mathrm{N}}

\DeclareMathOperator{\expct}{\mathrm{E}}

\DeclareMathOperator{\sgn}{\mathrm{sgn}}

\DeclareMathOperator{\dist}{\mathrm{d}}

\DeclareMathOperator{\nullspace}{\mathrm{N}}
\DeclareMathOperator{\range}{\mathrm{R}}

\def\transpose#1{{#1}^{\mathsf{T}}}
\def\minustranspose#1{{#1}^{\mathsf{-T}}}

\def\downsample#1{\downarrow_{#1}}
\def\upsample#1{\uparrow_{#1}}

%opening
\title{Digital Processing of Single and Multidimensional Signals}
\author{Aakash Jog}
\date{2018-19}

\begin{document}

\pagenumbering{roman}
\begin{titlepage}
\newgeometry{margin=0cm}
\maketitle
\end{titlepage}
\restoregeometry
%\setlength{\mathindent}{0pt}

\blfootnote
{
	\begin{figure}[H]
		\includegraphics[height = 12pt]{cc.pdf}
		\includegraphics[height = 12pt]{by.pdf}
		\includegraphics[height = 12pt]{nc.pdf}
		\includegraphics[height = 12pt]{sa.pdf}
	\end{figure}
	This work is licensed under the Creative Commons Attribution-NonCommercial-ShareAlike 4.0 International License. To view a copy of this license, visit \url{http://creativecommons.org/licenses/by-nc-sa/4.0/}.
} %CC-BY-NC-SA license

\tableofcontents

\clearpage
\section{Lecturer Information}

\textbf{Dr. Raja Giryes}\\
~\\
E-mail: \href{mailto:raja@tauex.tau.ac.il}{raja@tauex.tau.ac.il}\\
~\\

\clearpage
\pagenumbering{arabic}

\part{Basic Definition and Theorems}

\begin{definition}[Fourier transform]
	For an LTI system, the Fourier transform is defined as
	\begin{align*}
		\FT\left\{ x(t) \right\} &= \left\langle x,e^{2 \pi j f t} \right\rangle\\
		&= \int\limits_{t \in \mathrm{R}^d} x(t) \overline{e^{2 \pi j f t}} \dif t\\
		&= \int\limits_{t \in \mathrm{R}^d} x(t) e^{-2 \pi j \transpose{f} t} \dif t
	\end{align*}
	\label{def:Fourier_transform}
	\index{transform!Fourier!continuous time}
\end{definition}

\begin{definition}[Tensor product]
	\begin{align*}
		(x_1 \otimes \dots \otimes x_d)(t) &= x_1(t_1) \cdot \dots \cdot x_d(t_d)
	\end{align*}
	for $t \in R^d$.
	\index{product!tensor}
\end{definition}

\begin{definition}[Fourier transform of tensor product]
	The Fourier transform of a tensor product is defined as
	\begin{align*}
		\FT\left\{ x_1 \otimes \dots \otimes x_d \right\}(f) &= \int\limits_{R^d} x_1(t_1) \cdot \dots \cdot x_d(t_d) e^{-2 \pi j (t_1 f_1 + \dots + t_d f_d)} \dif t_1 \dots \dif t_d\\
		&= \left( \FT\left\{ x_1(t) \right\}(f_1) \right) \dots \left( \FT\left\{ x_d(t) \right\}(f_d) \right)
	\end{align*}
	\index{transform!Fourier!continuous time}
	\index{product!tensor}
\end{definition}

\begin{definition}[$\rect$]
	\begin{align*}
		\rect(t) &=
			\begin{cases}
				0 &;\quad |t| > \frac{1}{2}\\
				1 &;\quad |t| < \frac{1}{2}\\
			\end{cases}
	\end{align*}
	\index{standard functions!rect}
\end{definition}

\begin{definition}[$u_{[a,b]}$]
	\begin{align*}
		u_{[a,b]}(t) &=
			\begin{cases}
				1 &;\quad a \le t \le b\\
				0 &;\quad \text{otherwise}\\
			\end{cases}
	\end{align*}
	\index{standard functions!u@$u_{[a,b]}$}
\end{definition}

\begin{definition}[$\sinc$]
	\begin{align*}
		\sinc(t) &= \frac{\sin(\pi t)}{\pi t}
	\end{align*}
	\index{standard functions!sinc}
\end{definition}

\begin{theorem}
	\begin{align*}
		\FT\left\{ \rect(t) \right\} &= \sinc(f)
	\end{align*}
	\index{standard functions!rect}
	\index{standard functions!sinc}
\end{theorem}

\begin{definition}[$\boxfunc$]
	A box function is defined to be the product of rect functions in multiple dimensions.
	\begin{align*}
		\boxfunc(t_1,\dots,t_n) &= \rect(t_1) \cdot \dots \cdot \rect(t_n)\\
		&=
			\begin{cases}
				1 &;\quad |t_1| < \frac{1}{2}, \dots, |t_n| < \frac{1}{2}\\
				0 &;\quad \text{otherwise}\\
			\end{cases}
	\end{align*}
	\index{standard functions!box}
	\index{standard functions!rect}
\end{definition}

\begin{definition}[Signum function]
	The signum function is defined to be
	\begin{align*}
		\sgn(t) &=
			\begin{cases}
				1 &;\quad t > 0\\
				[-1,1] &;\quad t = 0\\
				-1 &;\quad t < 0\\
			\end{cases}
	\end{align*}
	\index{standard functions!sgn}
\end{definition}

\begin{theorem}[Shifting in time]
	\begin{align*}
		\FT\left\{ D_p\left( x(t) \right) \right\}(f) &= e^{-2 \pi j \transpose{f} p} \FT\left\{ x(t) \right\}(f)
	\end{align*}
	where
	\begin{align*}
		D_p\left( x(t) \right) &= x(t - p)
	\end{align*}
	\label{thm:shifting_in_time}
	\index{shift!in time}
\end{theorem}

\begin{theorem}
	\begin{align*}
		x(t) \ast h(t) &= \FT{x}(f) \FT{h}(f)
	\end{align*}
	\label{thm:convolution_in_time_multiplication_in_frequency}
	\index{convolution!in time}
	\index{multiplication!in frequency}
\end{theorem}

\begin{theorem}[Stretching in time]
	\begin{align*}
		\FT\left\{ S_A\left( x(t) \right) \right\}(f) &= \int\limits_{R^d} x(A t) e^{-j 2 \pi \transpose{f} t} \dif t\\
		&= \int\limits_{R^d} x(q) ^{-j 2 \pi \transpose{f} A^{-1} q} \frac{\dif q}{|\det A|}\\
		&= \frac{1}{|\det A|} \int\limits_{R^d}  x(q) e^{-j 2 \pi \left( \minustranspose{A} f \right)^T q} \dif q\\
		&= \frac{\FT\left\{ x(t) \right\}\left( \minustranspose{A} f \right)}{|\det A|}\\
		&= \frac{S_{\minustranspose{A}} X(f)}{|\det A|}
	\end{align*}
	where
	\begin{align*}
		S_A\left( x(t) \right) &= x(A t)
	\end{align*}
	where $A$ is a matrix which may represent stretching, rotating, etc.\\
	Hence, if $A$ is equal to a scalar $a$,
	\begin{align*}
		\FT\left\{ x(a t) \right\} &= \frac{\FT\left\{ x(t) \right\}\left( \frac{f}{a} \right)}{|a|}
	\end{align*}
	\label{thm:stretching_in_time}
	\index{stretch!in time}
\end{theorem}

\begin{theorem}[Fourier transform of Gaussian]
	Using \cref{thm:stretching_in_time}, for a single dimensional Gaussian signal,
	\begin{align*}
		\FT\left\{ e^{-t^2} \right\} &= e^{-\pi f^2}
	\end{align*}
	and for a orthogonal multidimensional Gaussian,
	\begin{align*}
		\FT\left\{ e^{-\transpose{t} t} \right\} &= e^{-\pi \transpose{f} f}
	\end{align*}
	Hence, using \cref{thm:stretching_in_time}, for a general multidimensional Gaussian,
	\begin{align*}
		\FT\left\{ e^{-\transpose{t} C^{-1} t} \right\} &= e^{-\pi \transpose{f} C f} |\det C|
	\end{align*}
	where $C$ is the covariance matrix.
	\label{thm:Fourier_transform_of_Gaussian}
	\index{transform!Fourier!continuous time}
	\index{standard functions!Gaussian}
\end{theorem}

\begin{definition}[Unitary matrix]
	A matrix $A$ is said to be unitary if
	\begin{align*}
		A^{-1} &= A^*
	\end{align*}
	where $A^*$ is the conjugate transpose of $A$.
	\index{types of matrices!unitary}
	\index{conjugate transpose}
\end{definition}

\begin{theorem}[Rotation]
	Stretching with by a unitary matrix $A$ is equivalent to rotation, and hence denoting $S_A$ by $R_R$,
	\begin{align*}
		\FT\left\{ R_R\left( x(t) \right) \right\} &= R_R \FT\left\{ x(t) \right\}
	\end{align*}
	\label{thm:rotation}
	\index{stretch!in time}
	\index{rotation!in time}
\end{theorem}

\begin{definition}
	A dimension-reducing projection $P$ from $R^d$ to $R^{d - 1}$, is defined as
	\begin{align*}
		P\left( x(t) \right) &= \int\limits_{R^d} x(t_1,\dots,t_d) \dif t_d
	\end{align*}
	\index{projection!dimension-reducing}
\end{definition}

\begin{definition}
	The slicing operator $\Q$ is defined as
	\begin{align*}
		\Q\left\{ X(f_1,\dots,f_d) \right\} &= X\left( f_1,\dots,f_{d - 1},0 \right)
	\end{align*}
	\label{def:slicing_operator}
	\index{standard operators!slicing}
\end{definition}

\begin{theorem}[Fourier transform of dimension reducing projection]
	The Fourier transform of a dimension reducing projection is
	\begin{align*}
		\FT\left\{ P\left( x(t) \right) \right\} &= \int\limits_{R^{d - 1}} \left( \int\limits_{R^d} x(t_1,\dots,t_d) \dif t_d \right) e^{-j 2 \pi (t_1 f_1 + \dots + t_{d - 1} f_{d - 1})} \dif t_1 \dots \dif t_{d - 1}\\
		&= \int\limits_{R^{d - 1}} \left( \int\limits_{R^d} x(t_1,\dots,t_d) e^{-j 2 \pi t_d (0)} \dif t_d \right) e^{-j 2 \pi (t_1 f_1 + \dots + t_{d - 1} f_{d - 1})} \dif t_1 \dots \dif t_{d - 1}\\
		&= \int\limits_{R^d} x(t) e^{-j 2 \pi \transpose{f} t} \dif t \Big|_{f_d = 0}\\
		&= \FT\left\{ x(t) \right\}(f_1,\dots,f_{d - 1},0)\\
		&= \Q\left\{ \FT\left\{ x(t) \right\} \right\}
	\end{align*}
	where $\Q$ is the slicing operator as in \cref{def:slicing_operator}.
	\label{thm:Fourier_transform_of_dimension_reducing_projection}
	\index{projection!dimension-reducing}
	\index{transform!Fourier!continuous time}
\end{theorem}

\clearpage
\part{Hilbert Spaces}

\section{Inner Product Spaces}

\begin{definition}[Inner product spaces]
	An inner product space is defined to be a space with an inner product with the following properties.
	\begin{description}
		\item[Conjugate symmetry]
			\begin{align*}
				\langle x,y \rangle &= \langle y,x \rangle^*
			\end{align*}
		\item[Linearity]
			\begin{align*}
				\langle x , a y + b z \rangle &= a \langle x,y \rangle + b \langle x,z \rangle
			\end{align*}
		\item[Non-negativity]
			\begin{equation*}
				\langle x,x \rangle = 0 \iff x = 0
			\end{equation*}
	\end{description}
	\index{spaces!inner product}
\end{definition}

\begin{definition}[Norm and distance]
	The norm corresponding to an inner product space is defined to be
	\begin{align*}
		\|x\| &= \sqrt{\langle x,x \rangle}
	\end{align*}
	Hence, the distance between $x$ and $y$ is defined to be
	\begin{align*}
		\dist(x,y) &= \|x - y\|
	\end{align*}
	\index{norm}
	\index{distance}
\end{definition}

\begin{definition}[Hilbert space]
	An inner product, normed, complete vector space is said to be a Hilbert space.
	\index{spaces!Hilbert}
\end{definition}

\section{Operators and Transformations}

\begin{theorem}
	Let $M$ be an operator from the inner product space $H$ to the inner product space $S$.
	Then,
	\begin{align*}
		\langle M x , y \rangle &= \langle x , M^* y \rangle
	\end{align*}
	for all $x \in H$ and $y \in S$.
	If $H$ and $S$ are finite, then $M$ can be represented by a matrix and
	\begin{align*}
		M^* &= \overline{M^T}
	\end{align*}
	is the complex transpose of $M$.
	\index{operators}
\end{theorem}

\begin{definition}[Linear operator]
	An operator $T$ is said to be linear if
	\begin{align*}
		T(a_1 x_1 + a_2 x_2) &= a_1 T(x_1) + a_2 T(x_2)
	\end{align*}
	\index{types of operators!linear}
\end{definition}

\begin{theorem}[Bounded linear operation]
	All linear operators are bounded, i.e. for every $T$, there exists $\alpha$ such that
	\begin{align*}
		\left\| T(x) \right\| &\le \alpha \|x\|
	\end{align*}
	for all $x$.
	\index{types of operators!linear}
\end{theorem}

\begin{theorem}[Continuous linear operation]
	If there exists a sequence $x_i$ which converges to $x$, then for any linear operator $T$, the sequence $T(x_i)$ converges to $T(x)$.
	\index{types of operators!linear}
\end{theorem}

\begin{definition}[Unitary operator]
	An operator $T$ is said to be unitary if and only if
	\begin{align*}
		T T^* &= T^* T\\
		&= I
	\end{align*}
	where $T^*$ is the conjugate transpose of $T$.
	\index{types of operators!unitary}
\end{definition}
%
\begin{definition}[Hermitian operator]
	An operator $T$ is said to be Hermitian if and only if
	\begin{align*}
		T^* &= T
	\end{align*}
	where $T^*$ is the conjugate transpose of $T$.
	\index{types of operators!Hermitian}
\end{definition}

\begin{theorem}
	An operator $T$ from a Hilbert space $H$ to $H$ is Hermitian if and only if
	\begin{align*}
		\langle T x , x \rangle &\in \mathbb{R}
	\end{align*}
	for all $x \in H$.
	\index{types of operators!Hermitian}
\end{theorem}

\begin{definition}[Orthogonal complement of subspace]
	The orthogonal complement of a subspace $W$ is defined to be
	\begin{align*}
		W^{\perp} &= \left\{ x \Big| \langle x,y \rangle = 0 \quad \forall y \in W \right\}
	\end{align*}
	\index{orthogonal complement}
\end{definition}

\begin{definition}[Null space of linear operator]
	The null space of a linear operator $T : H \to S$ is defined to be
	\begin{align*}
		\nullspace(T) &= \left\{ x \Big| T x = 0 \right\}\\
		&\subseteq H
	\end{align*}
	\index{null space}
\end{definition}

\begin{definition}[Range of linear operator]
	The range of a linear operator $T : H \to S$ is defined to be
	\begin{align*}
		\range(T) &= \left\{ y \Big| T x = y \, , \, x \in H \right\}\\
		&\subseteq S
	\end{align*}
	\index{range}
\end{definition}

\begin{definition}[Direct sum]
	The direct sum of two sets $A$ and $B$ is defined to be the set
	\begin{align*}
		C &= A \oplus B
	\end{align*}
	if and only if, for all $c \in C$, there exists a unique pair $a \in A$ and $b \in B$ such that
	\begin{align*}
		c &= a + b
	\end{align*}
	\index{direct sum}
\end{definition}

\begin{theorem}
	For a linear operator $T: H \to S$,
	\begin{align*}
		H &= \nullspace(T) \oplus \nullspace(T)^{\perp}
	\end{align*}
	where $\nullspace(T)$ is the null space of $T$.\\
	Additionally, if $S$ is a closed set,
	\begin{align*}
		S &= \range(T)^{C} \oplus \range(T)^{\perp}
	\end{align*}
	where
	\begin{align*}
		\range(T)^{C} &= \left( \range(T)^{\perp} \right)^{\perp}
	\end{align*}
	is the closure of $R(T)$.
	\index{null space}
	\index{closure}
\end{theorem}

\begin{definition}[Injective transformation]
	A linear transformation/operation $T: H \to S$ is said to be injective (one-to-one) if
	\begin{align*}
		x \neq y &\iff T x \neq T(y)
	\end{align*}
	\index{injectivity}
	\index{one-to-one}
\end{definition}

\begin{definition}[Surjective transformation]
	A linear transformation/operation $T: H \to S$ is said to be surjective (onto) if
	\begin{align*}
		R(T) &= S
	\end{align*}
	\index{surjectivity}
	\index{onto}
\end{definition}

\begin{definition}[Bijective transformation]
	A linear transformation/operation $T: H \to S$ is said to be bijective if it is injective (one-to-one) and surjective (onto).
	\index{bijectivity}
	\index{one-to-one and onto}
\end{definition}

\section{Singular Value Decomposition (SVD)}

\begin{definition}[Eigenvalue decomposition]
	The eigenvalue decomposition for a symmetric (and hence also square) and Hermitian matrix $A$ is
	\begin{align*}
		A &= V \Lambda V^*
	\end{align*}
	where $\Lambda$ is a diagonal matrix with the eigenvalues of $A$ as diagonal elements, and $V$ contains the corresponding eigenvectors of $A$.
	\index{decomposition!eigenvalue}
\end{definition}

\begin{definition}[Singular Value Decomposition (SVD)]
	Let $A \in \mathbb{R}^{m \times n}$.
	Then, the singular value decomposition of $A$ is defined to be
	\begin{align*}
		A &= U \Sigma V^*
	\end{align*}
	where the diagonal elements of $\Sigma$ are the singular values of $A$, and the remaining elements of $\Sigma$ are zero.\\
	The columns of $V$ are called the right singular vectors, the columns of $U$ are called the left singular vectors, and the diagonal elements of $\Sigma$ are denoted by $\sigma_i$, where $1 \le i \le \min(m,n)$.
	\index{decomposition!singular value}
\end{definition}

\begin{theorem}
	\begin{align*}
		{\sigma_i}^2 &= \lambda_i \left( A^* A \right)
	\end{align*}
	where $\sigma_i$ are the singular values of $A$, and $\lambda_i$ are the eigenvalues of $A$, for $1 \le i \le \min(m,n)$.
	\index{decomposition!eigenvalue}
	\index{decomposition!singular value}
\end{theorem}

\clearpage
\printindex

\end{document}
